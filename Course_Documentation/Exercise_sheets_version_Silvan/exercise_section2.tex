\newpage
\section{Exercise two: 2D data}
A radar image represents two-dimensional data. This requires not only focusing in range (see first exercise) but also focusing in azimuth (= along track, i.e. in the flight direction of the sensor). \\

You will find the formulas required for the range and azimuth chirp rates in the tutorial on epsilon nought (http://epsilon.nought.de/).

\subsection{2D Chirp (point source)}
Use the information in the file \textit{chirp\_2d\_test\_constants.txt} and \textit{chirp\_2d\_test.dat} to focus (or compress) the 2-D radar raw data. The image represent the raw-format of radar data of a single strong scatterer. After focusing in range an azimuth, the image still consists of complex values. This data format is called single-look-complex or SLC. Most radar data is distributed in the SLC format.

Definition of the .DAT format: two LONG values representing the number of pixels in range and azimuth direction. Then follows a sequence of complex values which represent a 2D matrix of the dimension given by range and azimuth. Write a binary-reader function to import the data. When you visualize the data, you should see several rings.
(IDL functions that may help: openr, readu, close, free\_lun, lonarr, complexarr, shade\_surf)

\subsubsection{compress the 2D raw data}
Plot intensity and phase of the compressed image. 
\subsubsection{compress the 2D raw data using an Hamming filter in azimuth}
Visualize the intensity and phase of the compressed image. 
\subsubsection{compress the 2D raw data using two looks in the azimuth spectrum}
Split the azimuth-spectrum and focus both spectra independently. Then average the magnitude of both images.
Literature: \textit{Chapter 2.2 in the "Scientific SAR User's Guide" by Coert Olmsted}.

\subsection{2D image from raw format}
Use the information in the file \textit{ers\_constants.txt} and the radar raw data \textit{ers\_raw\_demo.dat} to focus (or compress) the radar raw-data in the 2D chirp image. Note, that the image is negatively chirped ($Bw = -1.55404d7$) 
\begin{itemize}
\item Visualize the data after range compression
\item Visualize the data after azimuth compression
\item Visualize the data after azimuth compression with Hamming in azimuth
\end{itemize}
Hint: you can apply a spatial smoothing filter (boxcar, gauss, etc.) on the image intensity to remove the strong image speckle. This reduces the resolution but improves the radiometric accuracy of homogeneous areas. 
\subsection{Image interpretation}
Interpret the image. What can you see?
(tip: you might want to scale the amplitudes after applying the Hamming filter to conserve the overall power)
(IDL functions that may help: tv, bytscl, congrid, hanning)\\
Literature: \textit{Chapter 3.1 in the "Scientific SAR User's Guide" by Coert Olmsted}.


\subsection{2D image from range compressed format}
Use the information in the file \textit{rdemo040689\_cmp\_constants.txt} and \textit{rdemo040689\_cmp.dat} to focus (or compress) the 2D raw data. Note, that the data is already compressed in range.
\begin{itemize}
\item Plot data after azimuth compression (The data is ALREADY range compressed)
\item Plot data after azimuth compression AND applying a Hamming-window on the azimuth spectrum.
\end{itemize}
