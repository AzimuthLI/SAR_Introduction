\newpage
\section{Exercise five: PolInSAR}
The aim of the exercise is to understand how the random-volume over ground model works by  determination of the forest height according to the 3 stage inversion by  Cloude and Papathanassiou \cite{cloude03}. Data: sim\_rvog\_data.zip. This requires the following steps:
\begin{enumerate}
\item read the complex data. Plot amplitudes. Compute and plot coherence magnitudes and coherence phases for baseline 12 and baseline 13 (i.e. baseline12  = image 1 and image 2).
\item calculation of the Pauli components.
\item flat earth correction: Compute flat-earth phase analytically using the given constants and remove from slave pass. i.e. Determine geometric phase difference between pass 1 and pass 2 and pass 1 and pass 3 assuming flat topography. 
\item calculation of hh,vv, xx, pauli1 (HH+VV) and pauli2 (HH-VV) coherences.
\item plot magnitude and phase of coherence for hh, vv, xx (both baselines) in a 2D-plot.
\item compare in a histogram the magnitude of the coherence of hh, vv and xx (both baselines).
\item plot magnitude and phase of coherence for the five datasets hh, vv, xx, pauli1 and pauli2  (baseline 12) in the unit circle.
\item start 3-stage Pol-InSAR inversion for topographic phase, extinction, and tree height: Make a line fit (Linear fit to coherences. Plot complex unit circle and locations of 5 coherence points.  Fit line through the points.) HINT: use IDL LINFIT routine. To avoid infinite slopes one can also try shifting the points 90 degs. and re-compute LINFIT.  Take the solution with the lowest CHISQ error.
\item find intersection(s) between best-fit line and the unit circle and project the points onto the line. HINT: use math!  equation of a line and equation of a circle gives 2 eqns and two unknowns (coordinates of point of intersection)
\item determine the ground phase. HINT: Find Euclidean distance from each coherence to each of the two possible ground points.  Determine where the XX coherence (typically has a smaller ground contribution than the other coherences) lies in relation to the other coherences (= intersection point of line with circle that is further away from xx coherence). 
\item determine the ground phase (see 8 to 10) and show the ground phases of all points in a histogram.
\item create and plot lookup table (LUT) for coherence of vegetation: 
Assume XX coherence (projected to best-fit line) has no ground contribution. Perform integral from eqn. 8 (Cloude \& Papathanassiou 2003). Compute 2D LUT (look-up table) with different values of extinction (eg. vary from 0 to 2 dB/m) and height (eg. vary from 0 to 30 m or up to interferometry sensitivity height $h_\pi$).
\item calculate vegetation height and sigma (extinction) by the Look-up-table.
\item plot vegetation height histogram and 3D-plot (shade surf or contour).
\end{enumerate}
Please ask for help, if you are not advancing after half a day. You are not alone in this adventure. 

\begin{table}[h]
\footnotesize
\begin{tabular}{l@{= }lll}
H & 3e3 & (m) & sensor height\\
lambda & 0.24 & (m) & wavelength, L-band\\
$B_{12}$ & -10 & (m)  & horizontal baseline btween image 1 \& 2\\
$B_{13}$ & -20 & (m) & horizontal baseline between Image 1 \& 3\\
W  & 100e6] & (Hz) & bandwidth in range\\
grng\_res & 0.5 & (m) & ground range pixel spacing\\
$\theta_0$ & 45 & ($^\circ$)  & angle of incidence to image centre, assume constant for small area\\
$c$ & 3e8 & (m/s) & speed of light\\
$R_m$ & $H/\cos\theta_0$ & (m) & broadside range\\
$\alpha$ &  0  & ($^\circ$)& local slope\\
\end{tabular}
\caption{Constants for the calculation and of sim\_data\_rvog}
\end{table}
\noindent Hint: The data given are modelled. The input parameters are:\\
topographic phase = 0.0\\
extinction = 0. 1 dB/m\\
height = 10 m\\

\begin{thebibliography}{123}
\bibitem{cloude03} Cloude \& Papathanassiou "Three-stage inversion process for polarimetric SAR interferometry", 2003.
\bibitem{cloude98} Cloude \& Papathanassiou "Polarimetric SAR Interferometry",1998.
\bibitem{papa01} Papathanassiou \& Cloude "Single-Baseline Polarimetric SAR Interferometry", 2001.
\end{thebibliography}
